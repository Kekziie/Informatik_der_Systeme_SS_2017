\documentclass[paper=a4, fontsize=11pt]{scrartcl} 
\usepackage[utf8]{inputenc}
\usepackage{amsmath}
\usepackage{amsfonts}
\usepackage{amssymb}
\author{Kim Thuong Ngo}


\usepackage[T1]{fontenc} 
\usepackage{fourier} 

\usepackage{lipsum} 

\usepackage{listings}
\usepackage{graphicx}
\usepackage{tabularx}

\usepackage{sectsty}
\allsectionsfont{\centering \normalfont\scshape} 

\usepackage{fancyhdr} 
\pagestyle{fancyplain} 
\fancyhead{}
\fancyfoot[L]{} 
\fancyfoot[C]{} 
\fancyfoot[R]{\thepage} 
\renewcommand{\headrulewidth}{0pt} 
\renewcommand{\footrulewidth}{0pt}
\setlength{\headheight}{13.6pt}

\numberwithin{equation}{section} 
\numberwithin{figure}{section} 
\numberwithin{table}{section}

\setlength\parindent{0pt} 

\newcommand{\horrule}[1]{\rule{\linewidth}{#1}} 

\title{	
\normalfont \normalsize 
\textsc{Informatik der Systeme} \\ [25pt] 
\horrule{0.5pt} \\[0.4cm] 
\huge Aufgaben \\ 
\horrule{2pt} \\[0.5cm] 
}

\author{Kim Thuong Ngo} 

\date{\normalsize\today} 

%----------------------------------------------------------------------------------------

\begin{document}

\maketitle 

\newpage

\tableofcontents

\newpage

%----------------------------------------------------------------------------------------

\section{Zahlensysteme}

\section{Interpretation von Hexadezimalzahlen}

In den folgenden Tabelle sind in den ersten Spalten Zahlen im Hexadezimalsystem angegeben. Ergänzen Sie die folgende Tabelle von A)-E)! \\

\paragraph{a)}
Schreiben Sie die hexadezimale Zahl als binären String. Dabei soll der binäre String genauso viele Stellen besitzen, um jede zweistellige Hexadezimalzahl darstellen zu können. \\

\paragraph{b)}
Interpretieren Sie den binären String aus a) als vorzeichenlose Binärzahl. Geben Sie die Zahl im Dezimalsystem an. \\

\paragraph{c)}
Interpretieren Sie den binären String aus a) als Zweierkomplement. Geben Sie die Zahl im Dezimalsystem an. \\

\paragraph{d)}
Negieren Sie die Zahl aus c) und schreiben Sie diesen Wert als binären String im Zweierkomplement auf. \\

\paragraph{e)}
Schreiben Sie die Zahl aus c) als 16-stelligen binären String im Zweierkomplement aus. \\

\begin{tabular}{|c|c|c|c|c|c|}
\hline
& A) & B) & C) & D) & E) \\
\hline
0x20 & 0010 000$0_{2}$ & $32_{10}$ & $224_{10}$ & 1110 000$0_{2}$ & 0000 0000 0010 000$0_{2}$ \\
\hline
0xA2 & 1010 001$0_{2}$ & $162_{10}$ & $-94_{10}$ & 0101 111$0_{2}$ & 1111 1111 1010 001$0_{2}$ \\
\hline
0x9F & 1001 111$1_{2}$ & $159_{10}$ & $-97_{10}$ & 0110 000$1_{2}$ & 1111 1111 1001 111$1_{2}$ \\
\hline
0xC2 & 1100 001$0_{2}$ & $194_{10}$ & $-62_{10}$ & 0011 111$0_{2}$ & 1111 1111 1100 001$0_{2}$ \\
\hline
0x54 & 0101 010$0_{2}$ & $84_{10}$ & $172_{10}$ & 1010 110$0_{2}$ & 0000 0000 0101 010$0_{2}$ \\
\hline
0x92 & 1001 001$0_{2}$ & $146_{10}$ & $-110_{10}$ & 0110 111$0_{2}$ & 1111 1111 1001 001$0_{2}$ \\
\hline
0x3C & 0011 110$0_{2}$ & $60_{10}$ & $196_{10}$ & 1100 010$0_{2}$ & 0000 0000 0011 110$0_{2}$ \\
\hline
\end{tabular} \\

%--------------------------------------------------

\section{Umwandlung von Dezimalzahlen}

In der folgenden Tabelle sind in der ersten Spalte Zahlen im Dezimalsystem angegeben. Ergänzen Sie die folgende Tabelle von A)-C)!

\paragraph{a)}
Schreiben Sie den Absolutbetrag der Dezimalzahl als Hexadezimalzahl auf. \\

\paragraph{b)}
Schreiben Sie den Absolutbetrag der Dezimalzahl als binären String auf. \\

\paragraph{c)}
Schreiben Sie die Dezimalzahl als binären String im Zweierkomplement auf. \\

\begin{tabular}{|c|c|c|c|}
\hline
 & A) & B) & C) \\
\hline 
-24 & 0x18 & 0001 100$0_{2}$ & 1110 100$0_{2}$ \\
\hline
-119  & 0x77 & 0111 011$1_{2}$ & 1000 100$1_{2}$ \\
\hline
82   & 0x52 & 0101 001$0_{2}$ & 1010 111$0_{2}$ \\
\hline
125  & 0x7D & 0111 110$1_{2}$ & 1000 001$1_{2}$ \\
\hline
-17   & 0x19 & 0001 100$1_{2}$ & 1110 011$1_{2}$ \\
\hline
\end{tabular}

%----------------------------------------------------------------------------------------

\end{document}